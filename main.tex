%\documentclass[10pt,a4paper,twoside]{article}
\documentclass[10pt,a4paper]{article}
\usepackage{a4wide}
\linespread{1.359140914229522617680} 
\usepackage[utf8]{inputenc}
\usepackage[german]{babel}
\usepackage{amsmath}
\usepackage{amsfonts}
\usepackage{amssymb}
\usepackage{units} % soll ich das verwenden? oder siunitxs oder garnix?
\usepackage{hyperref}
\usepackage{ulem}
\hypersetup{
%    bookmarks=true,
    pdftoolbar=true,
    pdfmenubar=true,
    pdffitwindow=false,
    pdfstartview={FitH},
    pdftitle={HVDC},		% XXX: fix
    pdfauthor={Michael F. Sch\"onitzer},
    pdfsubject={HVDC},	% XXX: fix
    pdfkeywords={HVDC} {Strom} {Physik},	% XXX: fix
    pdfnewwindow=true,
    colorlinks=true,
    linkcolor=black,
    citecolor=black,
    filecolor=black,
    urlcolor=blue
}
\setcounter{secnumdepth}{4}

\newcommand{\degree}{$^\circ$}
\newcommand{\q}{\glqq }
\newcommand{\qe}{\grqq }
\newcommand{\rot}{\mathrm{rot\:}}

\usepackage{rotating}
\usepackage[basic]{circ}


\author{Michael F. Schönitzer}
\title{Physikalische Grundlagen der Energieübertragung mit Gleich- und Wechselstrom hoher Spannung }
\begin{document}
\maketitle

\section{Geschichtliches}

\section{Gleichstrom}
Die Energieübertragung mittels Gleichstrom ist im Grunde relativ einfach und bereits aus der Schule bekannt, wir wollen sie daher hier nur kurz zusammenfassen.
Eine Gleichspannungsquelle liefert einen zeitlich konstanten Potentialunterschied, elektrische Spannung $U$ genannt. Legt man an diese einen ohmschen Widerstand $R$ an, so fließt durch ihn eine zeitlich konstanter Strom
\begin{equation}
I = \frac{U}{R}
\end{equation}
Und leistet an ihm die Leistung
\begin{equation}
P = U \cdot I
\end{equation}



\section{Einphasen-Wechselstrom}\label{wechsel}
Beim Einphasen-Wechselstrom hat man einen Nullleiter und einen Leiter, auf dem eine sinusförmige Wechselspannung mit der Amplitude $U_0$ anliegt:
\begin{equation}
U(t)=U_0 \cdot \cos(\omega t + \varphi)
\end{equation}
Dieser Leiter wird in der Elektrotechnik oft umgangssprachlich als Phase bezeichnet; wir wollen davon hier jedoch nicht Gebrauch machen, da wir physikalisch korrekter den Faktor $\varphi$ als Phasenverschiebung oder kurz Phase bezeichnen werden.

Legt man obige Spannung an einen ohmschen Widerstand R an, so fließt durch ihn der Strom $I(t)= \frac{U(t)}{R} = I_0 \cdot \cos(\omega t + \varphi)$ mit $I_0 = U_0/R$.
Die elektrische Leistung $P$, die an dem Widerstand erbracht wird, wird analog zur Gleichspannungstechnik als Produkt aus Spannung und Strom definiert, dabei beziehen sich jedoch alle drei Größen auf die momentanen Werte:
\begin{equation}
P(t) = U(t) \cdot I(t) \stackrel{\mathrm{\varphi=0}}= U_0 I_0 \cdot \cos^2(\omega t)
\end{equation}
Berechnen wir nun den zeitlichen Mittelwert der Leistung
\begin{equation}\label{eq:Wirkleistung_ohne_phi}
\bar{P}=\frac1T \int\limits_0^T dt\: U_0 I_0 \cos^2 \omega t = \frac12 U_0 I_0 \qquad\mathrm{mit}\; T=2\pi/\omega
\end{equation}
Vergleicht man dies mit der Leistung eines Gleichstroms $P=U\cdot I$, so sieht man, dass eine Gleichspannung mit $U=U_0 / \sqrt2$ und dem daraus folgendem Strom $I=I_0 / \sqrt2$ dieselbe Leistung erbringt.
Deshalb definiert man die Effektivspannung und den Effektivstrom eines Wechselstroms als:
\begin{equation*}
U_{\mathrm{eff}} = \frac{U_0}{\sqrt2} \qquad I_{\mathrm{eff}} = \frac{I_0}{\sqrt2} \qquad \bar{P}=U_{\mathrm{eff}}I_{\mathrm{eff}}
\end{equation*}
Im folgenden werden wir mit $U$ beziehungsweise $I$ die Effektivwerte, mit $U(t)$ beziehungsweise $I(t)$ die Momentanwerte und mit $U_0$ beziehungsweise $I_0$ die Amplituden bezeichnen.


%%% INCLUDE-FILE: Spulen_Kondensatoren.tex %%%

\subsection{Spulen und Kondensatoren im Wechselstromkreis}

\subsubsection{Spulen in Wechselstromkreisen}
Wir betrachten nun einen Wechselstromkreis mit vernachlässigbaren Widerständen, in welcher sich eine Spule befindet. Ein Stromfluss in der Spule führt zu einem Magnetfeld. Da der Stromfluss in der Spule jedoch ständig sein Vorzeichen ändert, wird abwechselnd ein Magnetfeld in die eine Richtung aufgebaut und nach dem Zusammenbrechen dieses dann ein Magnetfeld mit umgekehrter Ausrichtung Richtung aufgebaut.
Die Summe aller Spannungen muss in einem Stromkreis immer Null sein:
$ U_Q + U_{ind} = 0 $.
Wobei $U_Q$ die von der Quelle angelegte Spannung ist und $U_ind$ die durch das Magnetfeld induzierte Spannung in der Spule ist. Letztere berechnet sich durch
\begin{equation}
U_{ind} = - L \cdot \frac{\mathrm dI(t)}{\mathrm dt}
\end{equation}
Daraus folgt durch einsetzen und integrieren
\begin{align}
%U_Q + U_{ind} =& 0 \nonumber \\
%\Rightarrow\;
U_0 \cos \omega t =& L \cdot \frac{\mathrm dI(t)}{\mathrm dt} \nonumber \\ 
\Rightarrow\; I(t) =& \frac{U_0}{L} \int \cos \omega t \mathrm dt = \nonumber \\ 
=& \frac{U_0}{\omega L} \sin \omega t = \nonumber\\
=& I_0 \sin \omega t \qquad mit\qquad I_0 = \frac{U_0}{\omega L}  \nonumber\\
=& I_0 \cos(\omega t - 90^\circ)
\end{align}
Der Strom hinkt der Spannung also um \unit{90}{\degree} hinterher und wir definieren analog zum ohmschen Widerstand den Betrag des induktiven Widerstand als den Quotient zwischen $U_0$ und $I_0$:
\begin{equation}
\left|Z_L\right| = \frac{U_0}{I_0} = \omega \cdot L
\end{equation}
Dies wird auch Induktanz $X_L = \left|Z_L\right| = \omega \cdot L$ genannt.

\subsubsection{Kondensatoren im Wechselstromkreis}
Analog zur Spule, können wir auch die Auswirkung eines Kondensators im stromkreis berechnen. %XXX "Analog"?
Wir beginnen dabei mit der Gleichung
\begin{equation}
U = \frac{Q}{C}
\end{equation}
für den Kondensator und differenzieren diese nach der Zeit:
\begin{equation}
\frac{dU}{dt} = \frac{1}{C}\frac{dQ}{dt} = \frac{1}{C} \cdot I
\end{equation}
Da die angelegte Spannung $U_Q$ der Spannung am Kondensator entspricht, gilt:
\begin{equation}
U_0\cdot \omega \sin \omega t = \frac{1}{C} \cdot I
\end{equation}
und somit
\begin{equation}
I = U_0\cdot \omega C \cos\left( \omega t + 90^\circ\right)
\end{equation}
Während bei einer Spule also der Strom der Spannung um \unit{90}{\degree} hinterher hinkt, eilt er bei einem Stromkreis mit Kondensator um \unit{90}{\degree} voraus. Wir ahnen bereits, dass diese beiden Effekte sich gegenseitig aufheben können.
Auch hier können wir analog zum ohmschen Widerstand den Betrag des kapazitiven Wiederstand definieren:
\begin{equation}
\left|Z_C\right| = \frac{U_0}{I_0} = \frac{1}{\omega C}
\end{equation}
Der Begriff Kapazitanz ist als $X_C = \left|Z_C\right| = -\frac{1}{\omega C}$ definiert.

\subsubsection{Allgemeiner Fall}
Betrachten wir nun einen Wechselstromkreis mit einer Induktivität, einer Kapazität und einem ohmschen Widerstand. Wieder gilt die 2. Kirchhoffsche Regel $\sum U = 0$:
\begin{equation}
U_Q = L \cdot \frac{\mathrm{d}I(t)}{\mathrm{d}t} + \frac{Q}{C} + I(t) \cdot R
\end{equation}
Wir differenzieren nach der Zeit um die Ladung los zu werden:
\begin{equation}
\frac{\mathrm dU_Q}{\mathrm dt} = L \cdot \frac{\mathrm{d^2}I(t)}{\mathrm{d}t^2} + \frac{I(t)}{C} + \frac{\mathrm dI(t)}{\mathrm dt} \cdot R
\end{equation}
Diese Differenzialgleichung können wir mit eines komplexem e-Ansatzes lösen:
Wir lösen diese Differenzialgleichung im Komplexen und betrachten dann den Realteil der komplexen Lösung gemäß Superpositionsprinzip als die physikalisch sinnvollen Lösungen. Zur Lösung  wählen wir den Ansatz:
\begin{equation}
U_Q = U_0 e^{(i\omega t)},	\qquad	I(t) = I_0 e^{i(\omega t-\varphi)}
\end{equation}
Die Wahl von U als komplexe Exponentialfunktion ist deshalb möglich, da der Realteil der komplexen Exponentialfunktion die Kosinus-Funktion ist.
\begin{equation}
Re\left[ U_0 e^{(i\omega t)}\right]  = U_0 \cos{(\omega t)}
\end{equation}
Durch einsetzen in den Ansatz erhalten wir:
\begin{equation}
i\omega U = (- L \omega^2 + i \omega R  + \frac{1}{C}) \cdot I(t)
\end{equation}
Wir definieren analog zum klassischen ohmschen Widerstand den Komplexen Widerstand als Quotient von Spannung und Strom:
\begin{equation}
Z = \frac{U}{I} = R + i ( \omega L - \frac{1}{\omega C})
\end{equation}
Wir stellen diese komplexe Größe nun in Polardarstellung:
\begin{equation}
Z = |Z| \cdot e^{i\varphi}
\end{equation}
mit
\begin{equation}\label{Zpolar}
|Z| = \sqrt{ R^2 + \left( \omega L - \frac{1}{\omega C} \right) }\quad {\mathrm{und}} \quad \tan\varphi = \frac{Im[Z]}{Re[Z]} = \frac{\omega L - \frac{1}{\omega C}}{R}
\end{equation}
Kehren wir nun zur Betrachtung im Reellen zurück um $I_0$ zu bestimmen:
\begin{align}
\nonumber
Re\left[ I \right]
&= Re\left[ \frac{I}{Z} \right] =\\\nonumber
&= Re\left[ \frac{U_0 e^{i\omega t}}{|Z|\cdot e^{i\varphi}} \right] =\\\nonumber
&= \frac{U_0}{|Z|} Re\left[ e^{i(\omega t - \varphi)} \right] =\\
&= \frac{U_0}{|Z|} \cos(\omega t - \varphi) = I_0 \cos(\omega t - \varphi)
\end{align}

Der Tangens der Phasenverschiebung also der Quotient aus Imaginärteil und Realteil des kompexen Widerstands, während die Amplitude der Stromkurve die Amplitude der Spannung durch den Betrag des kompexen Widerstands ist.
Man erkennt in \ref{Zpolar} leicht das die Phasenverschiebung null ist, wenn der Imaginärteil null ist, was für 
\begin{equation}
\omega L = \frac{1}{\omega C}
\end{equation}
der Fall ist. Bei richtiger Wahl von Induktivitäten bzw. Kapazitäten kann man also die Blindleistung einer Kapazität bzw. Induktivität auslöschen.



\subsection{Wirk-, Schein- und Blindleistung und der Leistungsfaktor}
Die Größen Spannung, Leistung und die Phasenverschiebung beschreiben das System vollständig, die Arbeit mit diesen Größen ist jedoch oft kompliziert und aufwändig. Daher zieht man häufig die Beschreibung mit den Größen Wirk-, Schein- und Blindleistung vor, weshalb wir diese hier definieren und näher betrachten wollen.

Die Wirkleistung ist diejenige Leistung, welche real genutzt werden kann, sie wird manchmal auch als aktive Leistung oder Nutzleistung bezeichnet. Sie ist das arithmetische Mittel über die Augenblicksleistung:
\begin{equation}
P = \overline{p} = \overline{U(t) \cdot I(t)}
\end{equation}
Für kosinusförmige Spannungs- und Stromverläufe wird daraus:
\begin{equation}
P = U_0 I_0 \cdot \frac{1}{2\pi} \int\limits_0^{2\pi} \cos\left( \omega t\right) \cos \left(\omega t - \varphi\right)
= \frac{1}{2} U_0 I_0 \cos \varphi
\end{equation}
Dies entspricht natürlich dem unter \eqref{eq:Wirkleistung_ohne_phi} Berechneten, jedoch nun mit Phasenverschiebung. Mit den Effektivwerten wird daraus also:
\begin{equation}
P =  U_{\mathrm{eff}} I_{\mathrm{eff}} \cdot \cos \varphi
\end{equation}
Die Wirkleistung wird in Watt (1 W = 1 V$\cdot$A) gemessen.

Die Scheinleistung ist hingegen das Produkt aus Spannung und Strom ohne Phasenfaktor. Die Verluste und die Beanspruchung der Komponenten verhalten sich, als würde eine Leistung dieser Größe übertragen. Sie stellt die bei einer gewissen Beanspruchung der Leitungskomponenten maximal übertragbare Wirkleistung dar. Sie ist deshalb immer größer oder gleich der Wirkleistung, wobei die Gleichheit für den Fall eintritt, dass die Phasenverschiebung null ist.
\begin{equation}
S = U_{\mathrm{eff}} I_{\mathrm{eff}} \geq P
\end{equation}
Die Scheinleistung ist zwar das Produkt aus Spannung (gemessen in Volt) und Strom (gemessen in Ampere), sie ist jedoch keine Leistung im physikalischen Sinn, deshalb wird sie nicht in Watt, sondern in Voltampere (VA) gemessen.

Als nächstes wollen wir den Leistungsfaktor einführen, er ist das Verhältnis von Wirk- und Scheinleistung:
\begin{equation}
\lambda = \frac{P}{S} \in [0,1]
\end{equation}
In unserem Fall betrachten wir kosinusförmige Spannungen und erhalten somit:
\begin{equation}
\lambda = \cos \varphi
\end{equation}
Wie im Folgenden leicht zu erkennen, beeinflusst der Leistungsfaktor den Wirkungsgrad der Komponenten:
\begin{equation}
\eta = \frac{P}{P+P_V} = \frac{\frac{P}{S}}{\frac{P}{S}+\frac{P_V}{S}} = \frac{\lambda}{\lambda + p_V} \qquad \mathrm{(mit}\;P_V\;\mathrm{als\: Verlustleistung)}
\end{equation}
Sinkt also der Leistungsfaktor, so reduziert sich dadurch auch der Wirkungsgrad einer Komponente.
Sofern nicht anders angegeben, bezieht sich der angegebene Wirkungsgrad einer Komponente immer auf einen Leistungsfaktor von 1, also der Situation, dass Strom und Spannung phasengleich verlaufen.
Hat man einen Phasenverschiebung ungleich null, so ist der Wirkungsgrad teils bedeutend geringer.
Für einen Leistungsfaktor von Null -- also eine Phasenverschiebung von 90\degree -- wird das System zwar von der Scheinleistung S beansprucht und es entstehen Verluste, es findet aber keinerlei Energieübertragung statt und der Wirkungsgrad ist null.
Der Leistungsfaktor kann auch als die Wurzel des Verhältnisses der minimal notwendigen und den tatsächlichen Verlusten ausgedrückt werden:
\begin{equation}
\lambda = \sqrt{\frac{P_{V\:min}}{P_V}}	% Herleitung?
\end{equation}
ein Leistungsfaktor von 0.8 würde also beispielsweise bedeuten, dass die Verluste 56\% größer sind als nötig.

Als dritte Leistungs-Größe wollen wir noch die Blindleistung $Q$ einführen.
%Diese Größe steht auf der Wirkleistung im Zeigerdiagramm orthogonal. %XXX Zeigerdiagramme hab ich noch nicht; Grafik?
%Sie lässt sich mit Pythagoras berechnen:
Die Blindleistung ist die Komponente der Scheinleistung, welche keine Arbeit verrichtet – also das gegenteil zur Wirkleistung. Sie ergibt sich im allgemeinen Fall aus $S^2=P^2+Q^2$ zu:
\begin{equation}
Q = \sqrt{S^2-P^2} = S \sqrt{1-\lambda^2} = P \frac{\sqrt{1-\lambda^2}}{\lambda}
\end{equation}
Für kosinusförmige Ströme wird sie zu:
\begin{equation}
Q = S \sqrt{1-\cos^2 \varphi} = S sin \varphi = U_{\mathrm{eff}} I_{\mathrm{eff}} sin \varphi
\end{equation}
Die Blindleistung trägt nicht zu Energieübertragung bei, sondern belastet nur die Komponenten. Sie ist ebenfalls keine physikalische Leistung, man misst sie in Voltampere reaktiv (Var\footnote{Var wird als Wort ausgesprochen.} oder älter VAr).
Die Blindleistung kann positiv oder negativ sein, per Konvention nimmt man sie als positiv an der Last an, wenn die Last induktiv ist. Dies ist vorteilhaft, da die meisten Lasten induktiv sind und man sie somit als Verbraucher von Wirk- und Blindleistung ansehen kann. Kondensatoren kann man als Quellen von Blindleistung ansehen.\cite{Harrison}

Zum Abschluss wollen wir noch die komplexe Leistung, auch Leistungsvektor oder komplexe Scheinleistung genannt, einführen: $\underline{S} = P+i Q$. Die oben eingeführte Scheinleistung $S$ ist dabei der Betrag der komplexen Scheinleistung $\underline{S}$.

% wo anders hin verschieben, da auch bei Gleichspannung!?
\subsection{Ohmscher Verlust}
Fließt ein Strom $I$ mit der Spannung $U$ durch einen ohmschen Widerstand, so wird dabei die elektrische Leistung
$P_V = U' \cdot I = R \cdot I^2$
in Wärme umgesetzt, wobei $U'$ die an dem Widerstand abfallende Spannung ist.
Da die übertragene Leistung $P=U \cdot I \cdot \cos(\varphi)$ ist, folgt für die Verlustleistung
\begin{equation}\label{eq:Verlust}
P_V = \left(\frac{P}{U \cdot \cos\varphi}\right)^2\cdot R
\end{equation}
Bei der Übertragung von Energie stellt das Kabel einen Widerstand dar und die Verlustleistung soll möglichst minimiert werden. Da der Widerstand nur bedingt verringert werden kann, erhöht man die Spannung und verringert somit den Stromfluss. Eine Leitung mit einer Spannung von 110\,kV hat einen Verlust von etwa 6\% je 100\,km, eine Leitung mit 800\,kV verliert auf der selben Distanz nur etwa 0.5\%.

Andererseits sehen wir, dass der Ohmsche Verlust mit steigender Phasenverschiebung ins Unendliche steigt. Daher ist man bemüht diese möglichst gering zu halten. Wie wir gesehen haben, ist dies durch Kompensation mit Spulen beziehungsweise Kondensatoren möglich.

Der Gleichstromwiederstand eines Materials hängt auch von der Temperatur $\vartheta$ des Leiters ab. Mann rechnet deshalb meist mit dem Widerstand $R_{20}$, den das Material bei einer Temperatur von $20^\circ C$ hat, und einem Temperaturbeiwert $\alpha_{20}$, der angibt wie stark der Widerstand mit der Temperatur steigt:
\begin{equation}
R_\vartheta = R_{20} \cdot \left[ 1 + \alpha_{20} \cdot \left( \vartheta - 20^\circ C \right) \right]
\end{equation}
Beide Werte sind empirisch bestimmte Materialkonstanten.

%%% INCLUDE-FILE: Skineffekt.tex %%%

\subsection{Skin-Effekt}
Der Widerstand einer Stromdurchflossen Leitung hängt neben der Länge und dem Material auch vom Querschnitt des Leiters ab. Dabei rechnet man in der Gleichstromtechnik mit
\begin{equation}
R = \rho \cdot l \cdot A
\end{equation}
Bei Wechselstrom kommt es jedoch zum sogenannten Skin-Effekt: der Stromfluss im Leiter wird nach außen verdrängt, es fließt also in den Äußeren Schichten wesentlich mehr Strom als in den inneren Schichten. Dieser Effekt ist um so stärker, je höher die Frequenz der Spannung ist. Bei den 50 Hz beziehungsweise 60 Hz die bei den Energienetzen üblich sind, ist der Effekt vergleichsweise schwach. Erst bei deutlich höheren Frequenzen wird der Effekt so stark, dass er der Stromfluss sich praktisch vollständig auf eine dünne Schicht (Haut) beschränkt -- woher der Effekt seinen Namen hat.

Die Stromdichte im Leiter nimmt nach außen hin gemäß
\begin{equation}
J = J_S e^{-\frac{d}{\rho}}
\end{equation}
ab, wobei $J_S$ die Stromdichte am Rand ist und die sogenannte äquivalenten Leitschichtdicke $\rho$ die Tiefe ist, in welcher die Stromdichte auf $1/e$ abgesunken ist. Für Kupfer beträgt der Wert von $\rho$ 9,38 mm (50 Hz), beziehungsweise 8,57 mm (60 Hz).

Der Skineffekt wurde 1873 von J. C. Maxwell vorhergesagt und 1885 von D. E. Hughe erstmals experimentell nachgewiesen\cite{BergmannSchaefer}.

Zum Verständniss des Effekts betrachten wir ein Flächenelement ds dr im Drahtinneren.
An dem Draht liegt eine Spannung an, welche also ein Elektrische Feld im Draht erzeugt
Andererseits fließt duch den Draht ein Strom, durch welchen ein Magnetische Feld aufgebaut wird.
Da es sich um Wechselstrom handelt, ändert das Magnetfeld ständig seine Richtung und Stärke, wodurch ein elektrisches Wirbelfeld induziert wird.
Auf der der Achse zugewandten Seite ist das induzierte Elektrische Feld dem äußerem Feld entgegengerichtet, auf der der Achse abgewandten Seite gleichgerichtet.
Das resultierende elektrische Feld und somit auch die Stromdichte muss also von innen nach ausen zunehmen.

Die genaue Herleitung des Skineffektes ist relativ kompliziert, Sie kann in einem Lehrbuch der Elektrodynamik nachgeschlagen werden. % zum beispiel…
Wir wollen hier lediglich eine einfache Herleitung der Eindringtiefe beschreiben.
Wir starten dabei mit den Maxwellgleichungen, genauer gesagt mit dem Induktionsgesetz von Faraday und dem erweiterten ampèreschen Gesetz:
\begin{align}
\nabla \times \mathbf{E} &= -\frac{\partial \mathbf{B}}{\partial t} \qquad &\mathrm{(Induktionsgesetz\ von\ Faraday)} \\
\nabla \times \mathbf{H} &= \mathbf{j} + \frac{\partial \mathbf{D}}{\partial t} \qquad &\mathrm{(Erweitertes\ amp\grave{e}resches\ Gesetz)}
\end{align}
Die Elektrische Flussdichte $D$ hängt mit der elektrischen Feldstärke $E$ zusammen:
\begin{equation}
\mathbf D = \epsilon_0 \epsilon_r \mathbf{E}
\end{equation}
Da die elektrisches Feldstärke von der Spannung abhängt, stellt auch sie eine Trigonometrische Funktion von $\omega t$ dar, die Ableitung wird somit zu:
\begin{equation}
\mathbf{\dot{D}} = \omega \epsilon_0 \mathbf{E}
\end{equation}\footnote{Warum wir hier das $\epsilon_r$ ignorieren können, habe ich leider nicht herausgefunden.}
Die Stromdichte j ergibt sich aus der Elektrischen Feldstärke und der elektrische Leitfähigkeit nach dem ohmschen Gesetz zu $j = \rho \mathbf E$. Daraus sehen wir, dass für $\omega \ll \frac{\rho}{\epsilon} \approx 10^{18} s^{-1}$ $\mathbf{\dot{D}}$ vernachlässigbar klein gegenüber $j$ ist.

Die magnetische Flussdichte $\mathbf B$ hängt analog mit der magnetischen Feldstärke $\mathbf H$ zusammen:
\begin{equation}
\mathbf B = \mu_0 \mu_r \mathbf H
\end{equation} 
Woraus folgt:
\begin{equation}
\mathbf{\dot{B}} = \mu_0 \mu_r \mathbf{\dot{H}}
\end{equation} 
Auf der anderen Seite folgt aus obiger Schreibweise des Ohmschen Gesetzes das 
\begin{equation}
\rot \mathbf E = \frac{1}{\rho} \rot j
\end{equation}
Damit folgt:
\begin{equation}
\rot \mathbf H = j \qquad
\rot \mathbf E = \frac{1}{\rho} \rot j = \mu_0 \mu_r \mathbf{\dot{H}}
\end{equation}
Woraus wird durch Elimination von $H$
\begin{equation}
\rot\rot j = \rho \mu_0 \mu_r \dot{j}
\end{equation}
Die zeitliche Ableitung entspricht einer Multiplikation mit $\omega$, die zweimalig räumliche ableitung (rot rot) einer zweimaligen Multiplikation mit der reziproken Schicht dicke, auf der der Strom auf $\frac{1}{e}$ abfällt, und wir erhalten:
\begin{equation}
\frac{1}{d^2} j \approx \omega \rho \mu_r \mu_0 j
\end{equation}



% wo anders hin verschieben, da auch bei Gleichspannung!?
\subsection{Koronaentladung} %QUELLEN
Liegt an einem vergleichsweise dünnem Kabel eine hohe Spannung an, so herrschen in seiner unmittelbaren Nähe sehr hohe elektrische Felder. Sind diese Feldstärken größenordnungsmäßig vergleichbar mit der Durchschlagfestigkeit von Luft (der Feldstärke ab der es zu einem Spannungsdurchschlag kommt), so wird die Luft ionisiert, was zu Verlusten führt. Bei Wechselspannung ist dieses Problem besonders groß, da die Scheitelspannung $U_0$ um den Faktor $\sqrt{2}$ größer als die Effektivspannung ist, es kommt also im Moment, wenn der Scheitelwert erreicht wird, auch bei geringeren Effektivspannungen bereits zur Koronaentladung. Dies betrifft insbesondere die 400-kV-Schiene (Scheitelspannung von 566 kV) und macht Übertragungen von mehr als 500 kV per Freileitung fast unmöglich.%Quelle, Richtig?

Die ist die Koronaentladung ist bei negativ geladen Leitern wesentlich höher als bei positiv geladenen\cite{Padiyar}, diese große Asymmetrie kann durch den Massenunterschied zwischen den Elektronen und den positiv geladenen Ionen. %Quelle

\section{Drehstrom}
Beim Dreiphasen-Wechselstrom, auch Drehstrom genannt, hat man drei stromführende Leitungen, an welchen je eine Kosinusförmige Spannung mit gleiche Frequenz, aber jeweils 120\degree Phasenverschiebung anliegt.
Die notwendigkeit eines zusätzlichen Nullleiters entfällt, das die drei Spannungen addiert sich aufheben.
Die Spannung zwischen einer beliebigen Leitung und einen Nullleiter wird als Sternspannung bezeichnet,
die Spannung zwischen zwei beliebigen Spannungen als verkettete Spannung. Es gilt:
\begin{equation}
U_{\mathrm{eff},\:\mathrm{verkettet}} = \sqrt{3} \cdot U_{\mathrm{eff},\:\mathrm{Stern}}
\end{equation}
Die Effekte und Aussagen aus Kapitel \ref{wechsel} gelten ansonsten analog auch für Drehstrom, weshalb wir sie nicht wiederholen wollten.

Während Privathaushalte nahezu ausnahmslos einphasigen Strom verwenden, verwendet man in der Industrie und vor allem in der Stromerzeugung und beim Stromtransport nahezu ausschließlich Dreiphasen-Wechselstrom. Dies bringt mehrere Vorteile\cite{Harrison}:
\begin{itemize}
\item Die durch Drehstrom übertragene Leistung ist zeitlich konstant, wodurch insbesondere Motoren ein zeitlich konstantes Drehmoment haben und somit ruhiger laufen.
\item Bei gegebener Größe hat eine dreiphasige Maschine eine höhere Leistung.
\item Bei gleichem Aufwand kann eine dreiphasiges Stromnetz mehr Leistung übertragen.
\end{itemize}

\section{Elektrische Netze}
\subsection{Leitungen}

\subsubsection{Leitungsbeläge}
Bisher haben wir die Grundlagen der Gleich- beziehungsweise Wechselstromtechnik im allgemeinen kennen gelernt.
Insbesondere bei der Betrachtung von Blindleistungen geht man dabei meist davon aus,
dass ein Verbraucher durch den Einsatz von Spulen oder Kondensatoren Blindleistung erzeugt.
Wir interessieren uns jedoch für die Energienetze und hier kommt ein wichtiger Punkt hinzu:
Auch die Leitungen selbst haben einen Widerstand, eine Kapazität, eine Induktivität, sowie die noch einzuführende Ableitung.
Diese bezeichnet man auf die Länge bezogen als Widerstand-, Kapazitäts-, Induktivitäts- und Ableitungsbeläge und fasst sie unter dem Begriff der Leitungsbeläge zusammen.

Zunächst sei noch die Ableitung eingeführt: Die Ableitungsbeläge beschreiben die Verluste durch unvollständige Isolation des Kabels. Die Ableitung wird in Siemens\footnote{Eine alternative, ältere Bezeichnung für das Siemens ist das Mho (Ohm Rückwerts gelesen) $\mho$ } gemessen, was den Kehrwert der Einheit Ohm darstellt:
\begin{equation}
\mathrm{[S]} = \frac{1}{\Omega} = \frac{A}{V}
\end{equation}
Die Ableitungsbeläge folglich in Siemens pro Meter.
Die Ableitung ist dank moderner Isolationstechnik deutlich geringer als die Ohmschen Verluste.

Bei der Betrachtung der Leitungsbeläge und der daraus folgenden Anforderungen an die Leitungen muss man zwischen drei unterschiedlichen Leitungstypen unterscheiden: Freileitungen, Erdkabel und Unterseekabel %  Begriff richtig?

\subsubsection{Freileitungen}
Der überwiegende Teil der Hochspannungs-Energieverteilung geschieht mit Freileitungen.
Dabei werden nicht isolierte Leiterseile an Isolatoren an etwa 50 Meter hohen und etwa 400 Meter voneinander entfernten Hochspannungsmasten aufgehängt. % Werte für GB
In Deutschland gibt es Freileitungen mit bis zu 380 kV in anderen Ländern auch bis 1000 kV und Anlagen mit bis zu 2 MV sin geplant\cite{Flosdorff}.
Es gibt unterschiedliche Arten und Bauweisen von Freileitungsmasten, auf die wir hier nicht näher eingehen wollen. Wichtig sind für uns lediglich die sich dadurch ergebende Anzahl und Anordnung der Leiterseile, deren Bodenhöhe und Abstand zu einander.

Die Spitze der Masten tragen ein Erdseil, welches die Leiterseile vor Blitzschlag schützen, indem Sie ein Schutzraum bilden. Er wird näherungsweise durch den Zwischenraum zwischen zwei den Boden und das Erdseil tangierenden Kreisen mit doppelter Masthöhe als Radius beschreiben. Die Masten sind derartig gebaut, dass die die Leiterseile sich in diesem Schutzraum befinden. Um bei – bei besonders hohen Spannungen nötige – weiten Travernen den Mast nicht zu hoch bauen zu müssen, kann man durch Verwendung von zwei statt nur einem Erdseil den Schutzraum vergrößern.

Bei den Leiterseilen handelt es sich meist um Verbundseile mit einem Kern aus Stahl, der die mechanische Stabilität gewährleistet, umgeben von Aluminiumleitern. Aluminium hat zwar im Bezug auf das Volumen einen höheren Widerstand als Kupfer, im Bezug auf das Gewicht leitet Aluminium hingegen besser. Ein typisches 650-A-Leiterseil besteht aus 7 Stahl- und 54 Aluminiumadern, die jeweils einen Durchmesser von 3 mm haben.\cite{Harrison}.
Moderne Leiterseile bestehen oft ganz aus einer Aluminiumlegierung und haben dadurch einen höheren Stromtragefähigkeit, einen geringen ohmschen Widerstand und müssen weniger gewartet werden\cite{Harrison}.
%% Skineffekt
Bei höheren Spannungen verwendet man statt einer dicken mehrere dünnere Leiter pro Phase seil in geringem Abstand. Dadurch sind die Randfeldstärken\footnote{Die Randfeldstärke darf zur Vermeidung von Koronaentladung nicht größer als 17 kV/cm werden.\cite{Flosdorff}} und die Induktivität geringer und durch die bessere Kühlung aufgrund der größeren Oberfläche kann ein größerer Strom geführt werden. Man spricht hier von Bündelleitern.
Seit einigen Jahren stattet man vor allem die Erdseile, aber auch die Leiterseile mit Lichtwellenleitern zur Datenübertragung aus. Diese dienten zunächst der innerbetrieblichen Fernüberwachung- und -steuerung, wird heute jedoch vor allem auch für privat genutzte Telefonnetze verwendet.\cite{Flosdorff}
% Durchhang, entfernung, Last, Wetter, etc. ?

In Abbildung \ref{fehlt} ist ein Ersatzschaltbild für eine Einphasige Leitung mit den Leitungsbelägen gezeichnet. Dabei ist R der serielle Widerstand, L die serielle Induktivität, C die Parallele Kapazität und $R_0$ der Ableitungswiederstand. Für die genaue Betrachtung muss man sich dieses infinitesimal Klein, unendlich oft hintereinander geschaltet vorstellen, für kurze Leitungen (unter 100 km Länge), kann man es jedoch näherungsweise auf eine derartige Schaltung reduzieren. Außerdem sind im allgemeinen die Induktivitäten nicht notwendigerweise zeitlich konstant.\cite{Flosdorff}
Typische Werte für den Widerstand, die Induktivität und Kapazität findet man in Tabelle \ref{fehlt}. Der Wert für die Ableitungsbelag hängt stark von äußeren Faktoren, wie den Umgebungsbedingungen und dem Verschmutzungsgrad der Isolatoren ab. Ein typische Wert ist $200\,M\Omega m^{-1}$.\cite{Harrison} Die Ableitungsverluste sind also meist deutlich geringer als die seriellen ohmschen Verluste, weshalb man sie in den meisten Betrachtungen ignorieren kann. Wie wir sehen ist eine Freileitung weitgehend induktiv.

%%% Induktivität %%%
\paragraph{Induktivitätsbelag}
Fließen Wechselströme durch mehrere nebeneinander liegende Leiter so wird durch Selbstinduktion und Gegeninduktion eine Wechselspannung induziert.
%bla, von wegen kan als selbstinduktion gesehen werden
Die Induktivität eines Leiters $v$ hängt wie folgt mit dm magnetischem Fluss an dessen Position zusammen:
\begin{equation}
L_v = N_v \frac{\Phi_v}{i_v}
\end{equation}
Laut \cite{Flosdorff} % besser [42] dort
ist $N_v = 1$, somit müssen wir um die Induktivität zu berechnen den magnetischen Fluss berechnen, welchen sich aus den Flüssen aller Leiter zusammensetzt. Es gilt also das Gleichungssystem:
\begin{equation}\label{SummePhi}
\Phi_v = \sum_k \Phi_{vk}
\end{equation}
%Da Feldlinien stets von einer Quelle ausgehen und in einer Senke enden, benötigen wir noch eine äußere Feldbegrenzung. %realy?
Zur Berechnung der Flüsse führen wir einen Zylinder mit sehr großem, aber endlichem Radius $r_a$ und den Leiter $v$ als Mittelpunkt als Feldbegrenzung ein. Die Rechtfertigung dafür werden wir später sehen.
Der Fluss den der Leiter selbst erzeugt ist nach \ref{42}
\begin{equation}
\Phi_{vv} = \frac{\mu_0l}{2\pi} \left( \ln\frac{r_a}{r_v} + 0,25 \right) i_v
\end{equation}
Der Fluss im Leiter $v$ der von einem Leiter $k$ erzeugt wird ist das Integral des Magnetfelds über die Fläche zwischen Leiter und Begrenzungszylinder:
\begin{equation}
\Phi_{vk} = \int_A B_k dA = \int\limits_{x=d_{vk}}^{x=r_a} \frac{\mu_0i_k l}{2\pi x}dx =
\frac{\mu_0l}{2\pi}\ln\left(\frac{r_a}{d_{vk}}\right) i_k
\end{equation}
Setzt man dies in Gleichung \ref{SummePhi} ein, so erhält man:
\begin{equation}
\Phi_v = \frac{\mu_0l}{2\pi} \left[ \left( \ln\frac{r_a}{r_v} +0,25\right) i_v + \sum_{k\neq1}\ln\frac{r_a}{d_{vk}} i_k \right]
\end{equation}
Wir beziehen den Fluss auf die Länge der Leitung und zeihen $r_a$ raus:
\begin{equation}
\Phi'_v = \frac{\mu_0}{2\pi}
 \left[
   \ln r_a \sum i_k +
   \sum_k \left( \ln\frac{1}{d_{vk}} i_k +
   0,25 \: \delta_{ik} \right)
 \right]
\quad mit \: d_{v1}:=r_v
\end{equation}
Die Summe aller Leiterströme muss immer null sein $\sum i_k=0$ und somit verschwindet der erste Term und die Gleichung ist unabhängig von $r_a$. Das Gleichungssystem für die Flüsse wird somit zu:
\begin{equation}
\Phi'_v = \sum_k a_{vk}i_k
\end{equation}
mit den Koeffizienten:
\begin{equation}
a_vv = \frac{\mu_0}{2\pi} \left( \ln\frac{1}{r_v} + 0,25 \right) \qquad und \qquad a_{vk} = a_{kv} = \frac{\mu_0}{2\pi} \ln\frac{1}{d_{ik}}
\end{equation}
Betrachten wir zunächst ein \textbf{Zweileitersystem}. Das obige Gleichungssystem reduziert sich auf:
\begin{align}
\Phi'_1 &= a_{11}i_1 + a_{12}i_2 \\
\Phi'_2 &= a_{21}i_i + a_{22}i_2
\end{align}
Dabei ist $i_2 = -i_1$ und $d_{12} = d_{21} = d$. Die Induktivität errechnet sich nun einfach zu:
\begin{equation}
L'_1 = a_{11} - a{12} = \frac{\mu_0}{2\pi}\left(\ln\frac{1}{r_1}+025\right)-\frac{\mu_0}{2\pi}\ln\frac{1}{d} = 
\frac{\mu_0}{2\pi}\left(\ln\frac{d}{r_1}+025\right)
\end{equation}
und gleichfalls für Leiter 2. In der Regel sind die beiden Leiterradien gleich, wodurch beide Leiter die selbe Induktivität haben – was aus Symmetriegründen selbst verständlich war. Will man die Induktivität nicht auf die Leiterlänge sonder auf die Leitungslänge beziehen, so muss man die Induktivität verdoppeln, da man zwei Leiter hat – man erhält also schließlich:
\begin{equation}
L' = \frac{\mu_0}{\pi}\left(\ln\frac{d}{r_1}+025\right)
\end{equation}

Ein weiterer wichtiger Sonderfall ist die symmetrische Dreileiteranordnung, bei welcher alle drei Leiterabstände sowie Leiterradien jeweils gleichgroß sind: $d_1=d_2=d_3:=d$ und $r_1=r_2=r_3:=r$ und somit $a_{12}=a–{23}=a_{13}$. Auch hier sind die Induktivitäten der Leiter gleich groß. Wir errechnen für den ersten Leiter:
\begin{equation}
\Phi'_1 = a_{11}i_1+a_{12}i_2+a_{13}i_3 = a_{11}i_1+a_{12}\left(i_2+i_3\right) = \left(a_{11}-a_{12}\right)i_1
\end{equation}
und daraus:
\begin{equation}
L'_1 = \frac{\Phi'_1}{i_1} = a_{11}-a_{12} = \frac{\mu_0}{2\pi}\left(\ln\frac{1}{r}+0,25-\ln\frac{1}{d} \right) =
\frac{\mu_0}{2\pi}\left(\ln\frac{d}{r}+0,25 \right)
\end{equation}
Ist das Drehstromnetz gleichphasig belastet, so addieren sich die Spannungen und Ströme zu null und es genügt einphasig zu rechnen, da keine Rückführung mehr nötig ist. Die auf die Leitungslänge gerechnete Induktivität ist also:
\begin{equation}\label{Induktivitaet3}
L' = L'_1 = \frac{\mu_0}{2\pi}\left(\ln\frac{d}{r}+0,25 \right)
\end{equation}
Laut \cite{Harrison} gilt diese Gleichung nur für $h\gg d$ – ist dies nicht gegeben, muss man Einflüsse des Bodens berücksichtigen.

Sind die Leiter nicht symmetrisch mit gleichen Leiterabständen angeordnet, so ergeben sich für die unterschiedlichen Leiter unterschiedliche Induktivitäten. Da dies im Allgemeinen unerwünscht ist, verdrillt man die Leiter: man wechselt die Positionen periodisch, so das jedes Kabel einmal jede der drei Positionen eingenommen hat. Dadurch gleichen sich die unterschiedlichen Induktivitäten in der Leiter aus und es gilt Gleichung (\ref{Induktivitaet3}) mit dem geometrischem Mittelwert der Leiterabstände
\begin{equation}
\bar{d} = \sqrt[3]{d_{12}d_{23}d_{31}}
\end{equation}
anstatt $d$.
%% Vereinfachung
%% Formel für Last

%%% Berechnung Kapazität %%%
\paragraph{Kapazitätsbelag}
 Die Kapazität einer Freileitung entsteht durch die die unterschiedlichen Potentiale der Leiterseil gegeneinander und gegenüber dem Erdpotential. Die Potentialunterschiede führen zu einem Elektrischem Feld, bestehend aus den elektrischen Flüssen $\Psi_{ij}$ zwischen zwei Leiterseilen, sowie den Flüssen $\Psi_{i0}$ zwischen je einem Leiter und dem Erdpotential. Zu jedem dieser Elektrischen Flüsse gehört eine Kapazität (Teilkapazität), welche zusammen die Kapazität der Freileitung ergeben.
Wir betrachten zunächst nur einen Leiter. Zur Berechnung der Kapazität verwenden wir Spiegelladungen: Wir ersetzen den Grund durch eine an der Erdoberfläche gespiegelte Ladung mit entgegengesetzter Ladung $Q'_1 = - Q_1$, ohne das sich das Feld ändert. Nun können wir das Potential in einem beliebigen Punkt P Als Summe der Potentiale schreiben:
\begin{equation}
\varphi = \frac{Q_1}{2\pi\varepsilon_0} \int^{x_1}_{r_1} \frac{dx_1}{x_1} + \varphi_1 + \frac{Q'_1}{2\pi\varepsilon_0} \int^{x_1}_{r_1} \frac{dx'_1}{x'_1} + \varphi'_1
\end{equation}
Wobei $r_1$ der Radius der Leiter ist, welche wir im folgenden als klein im Vergleich zu den Leiterabständen betrachten. Mit $Q'_1 = - Q_1$ und $\varphi'_1 = - \varphi_1$ ergibt sich:
\begin{equation}\label{einleitungsfeld}
\varphi = \frac{Q_1}{2\pi l\varepsilon_0} \ln \frac{x'_1}{x_1}
\end{equation}
An der Oberfläche des Leiters ist das Potential somit näherungsweise:
\begin{equation}
\varphi^* = \frac{Q_1}{2\pi l\varepsilon_0} \ln \frac{2h_1}{r_1}
\end{equation}
Dies entspricht aus auf Grund der Stetigkeit des Potentials % echt?
dem Potential des Leiters $\varphi_1 = \varphi^*$.
Betrachten wir nun ein System mit mehreren Leitern. Das Potential an einem wieder frei gewähltem Punkt P berechnet sich als die Summe der nach (\ref{einleitungsfeld}) berechneten Potentiale:
\begin{equation}
\varphi = \sum \frac{Q_i}{2\pi l\varepsilon_0} \ln \frac{x'i}{x_i}
\end{equation}
Wie zuvor setzen wir die Ortskoordinaten der Leiteoberflächen ein um die Potentiale der Leiter zu bestimmen, dadurch erhalten wir:
\begin{align}
\varphi_1 &= a_{11} Q_1 + a_{12} Q_{2} + ... + a_{1n} Q_n \\
\varphi_2 &= a_{21} Q_1 + a_{22} Q_{2} + ... + a_{2n} Q_n \\
\vdots \\
\varphi_n &= a_{n1} Q_1 + a_{n2} Q_{2} + ... + a_{nn} Q_n
\end{align}
mit den Koeffizienten
\begin{equation}
a_{ii} = \frac{\ln 2h_i/r_i}{2\pi l \varepsilon_0} \quad \textrm{und} \quad a_{ik} = \frac{\ln D_{ik}/d_{ik}}{2\pi l \varepsilon_0}
\end{equation}
Das obige Gleichungssystem muss nun nach den $Q_i$ aufgelöst werden. Das Gleichungssystem lässt sich auch in Matrizenschreibweiße $\boldsymbol{\varphi} = \uuline{A} \cdot \mathbf{Q}$ darstellen, dann entspricht das Auflösen der Invertierung der Matrix $\uuline{A}$: $\uuline{D}=\uuline{A}^{-1}$. Man erhält also allgemein:
\begin{equation}
Q_i = \sum_k d_{ik}\varphi_k
\end{equation}
Wir ziehen das Element mit $k=i$ aus der Summe hinaus und ergänzen:
\begin{equation}
Q_i = \sum_{k\neq i} d_{ik}\varphi_k + d_{ii}\varphi_i + \textcolor{blue}{\sum_k d_{ik}\varphi_i} - \textcolor{blue}
{\sum_k d_{ik}\varphi_i}
\end{equation}
Durch umsortieren und ausklammern erhalten wir schließlich eine Form in welcher wir die Teilkapazitäten identifizieren können.
\begin{equation}
Q_i = \underbrace{\left(d_{ii}+\sum_{k\neq i}d_{ik}\right)}_{:=C_{i0}}\varphi_i + \sum_{k\neq i} \underbrace{d_{ik}}_{:=C_{ik}} \left( \varphi_k - \varphi_i \right)
\end{equation}
Mithilfe der Regeln zur Parallel- und Reihenschaltung von Kondensatoren lässt sich daraus der Gesamtkapazitätsbelag berechnen.
Für eine Wechselstromleitung mit zwei Leiterseilen erhält man gemäß dieser Vorgehensweise\footnote{Dabei schrumpft das Gleichungssystem jedoch auf zwei Gleichungen zusammen, weshalb die Auflösung sehr einfach wird.}
\begin{equation}
C' = \frac{C}{l} = \frac{\varepsilon_0\pi}{\ln{\frac{2hd}{r\sqrt{\left(2h\right)^2+d^2}}}}
\end{equation}
Für eine Dreiphasenleitung ohne Erdleiter kann man man die Herleitung vereinfachen indem man ausnützt, dass die Ladungen der drei Leitungen zusammen immer Null ergeben. Möchte man jedoch, z. Bsp. für  Erdschlussstromberechnungen nicht nur die Gesamtkapazität sondern auch die einzelnen Teilkapazitäten berechnen, geht man nach dem beschriebenen Ansatz vor.
Da die Koeffizienten $a_{ii}$ und somit auch die Koeffizienten $d_{ik}$ von der Höhe der Leitungen über dem Erdboden abhängen, ergibt sich -- im Gegensatz zur Induktivität -- bei einer symmetrischen Dreiecksanordnung wie oben %verwais
unterschiedliche Kapazitäten für die Leitungen. Daher verdrillt man auch derartige symmetrische Leiter. Für einen derartigen Leiter erhält man den Kapazitätsbelag
\begin{equation}\label{verdrillteindukt}
C' = \frac{2\pi\varepsilon_0}{\ln\left(2 \bar{h}\bar{d}/(r\bar{D})\right) } \approx
\frac{2\pi\varepsilon_0}{\ln\left(\bar{d}/r\right)}
\end{equation}
Wobei $\bar{h}$, $\bar{d}$ und $\bar{D}$ die geometrischen Mittelwerte der Leiterhöhen, -Abstände und der Abstände zwischen einer Leitung und einer anderen \q Spiegelleitung\q : % Anführungszeichen fixen
\begin{equation}
\bar{h} := \sqrt[3]{h_1h_2h_3}, \quad \bar{d} := \sqrt[3]{d_{12}d_{23}d_{31}} \quad \bar{D} := \sqrt[3]{d_{12'}d_{23'}d_{31'}} \approx 2\bar{h}
\end{equation}
Man kann zeigen, dass ein zusätzliches Erdseil die Erdkapazität erhöht und die die Leiterkapazität verringert, so dass die Gesammtkapaziät unverändert bleibt und Gleichung (\ref{verdrillteindukt}) bleibt gültig.\cite{Flosdorff}


\subsubsection{Kabel}
In urbanen Gegenden sowie in Gegenden mit besonders erhaltenswerten Landschaften, setzt man statt Freileitungen auf unter der Erde verlegt Kabel. Die Leiter sind in der Regel aus Kupfer oder Aluminium und mit Ölimpregniertem Ölband oder neuer mit speziellen Kunststoffen isoliert. % flüssigkeitsimprägniertem Polypropylen/Papier-Laminat isoliert.
Neben Kosten und Gewichtseinsparungen haben moderne Kunststoffisolierungen auch elektrische Vorteile, so haben Kabel mit flüssigkeitsimprägniertem Polypropylen/Papier-Laminat eine um 67\% niedrigere dielektrische Verluste und eine 20\% niedrigere Permittivität und somit Kapazität.\cite{Harrison} Dadurch lässt sich mehr Leistung übertragen und es muss weniger Blindleistung bereitgestellt werden.
%Gaskablenm Druckgaskabel, (Druck-)Ölkabel
PVC-Kabel werden in Niederspannungs- und Mittelspannungsnetzen bis 10 kV eingesetzt. Für Hochspannungsnetze ist es wegen der hohen dielektischen Verluste ungeeignet, hier kommen neben Ölkabeln und Druckgaskabeln neuerdings Kabel mit vernetztem Polyäthylen (VPE) zum Einsatz. VPE hat hervorragende dielektrische und thermische Eigenschaften, weshalb es immer mehr in den Mittel und Niederspannungsbereich vordringt.\cite{Flosdorff}

% höhere Kosten, nidrigere Zuverlässigkeit
Das Ersatzschaltbild sieht wie das der Freileitung aus, nur das man zwei Parallel geschaltete Ableitungswiederstände betrachtet.
$R_0$ ist der Isolationswiederstand, während $R_d$ die dielektrischen Verluste beschreibt. Das sind Verluste die durch Polarisationseffekte im Dielektrikum der Kapazität entstehen.
Die Induktivitäten von Kabeln lassen sich nach den selben Gleichungen berechnen, wie die von Freileitungen, da hier Herleitung dieser ohne die Bedingung $d\gg r$ auskommt.\footnote{Dabei ist $\mu_0$ natürlich durch $\mu_r\mu_0$ – mit $\mu_r$ als der relativen Permeabilität des Materials – zu ersetzen} %was ist mit h>>d ?
Da bei Kabel die Leiterabstände deutlich kleiner sind sind auch die Induktivitäten kleiner: sie betragen nur etwas 25\% bis 30\% derWerte von Freileitungen\cite{Flosdorff}.
Die Berechnung der Kapazität von Kabeln, insbesondere Gürtelkabeln, ist ungleich schwieriger als bei Freileitern, da wir vorausgesetzt haben, dass $d\gg r$ ist auserdem sind die Dielektriken und der kreisförmige Metallmantel rechnerisch nur näherungsweise bestimmbar. Daher werden die Kapazitäten experimentell vom Hersteller bestimmt.\cite{Flosdorff}
Typische Werte für ein einphasiges Kupferkabel findet man in Tabelle \ref{fehlt}.
Wie wir mit den Werten feststellen können, ist bei Kabeln die Kapazität überwiegend,
besonders stark ist dies bei Unterseekabeln ausgeprägt. % überprüfen, warum?, Werte
Dazu kommt, dass bei Unterseekabeln die Blindleistungskompensation nur schwer möglich ist.
Die hohen Kapazitäten erfordern hohe Ströme zum Laden der Kapazität (Ladeströme), welche durch den gesamten zur Verfügung gestellten Strom aufgebracht werden müssen. Dies reduziert die übertragbare Leistung und führt ab einer gewissen Länge dazu, dass der maximal übertragbare Strom als Ladekapazität benötigt wird. % versteh ich selbst nur halb...
Für ein 400-kV-Kabel wie in Tabelle \ref{fehlt} liegt diese Länge bei 43,6 km.
Daher werden bei längeren Kabeln alle paar Kilometer % wieviel genau, besser formulieren…
Induktivitäten angeschlossen um die Blindleistung zu kompensieren.

\section{Netzregelung}
Elektrische Netze leiten elektrische Energie, können aber weder Energie erzeugen noch speichern.
Aufgrund der Energieerhaltung muss zu jedem Zeitpunkt exakt die gleiche Menge an Wirk- und Blindleistung in das Netz eingespeist werden, wie – einschließlich aller Verluste – verbraucht wird.
Es stellt sich daher die Frage, wie dies gewährleistet werden kann und was geschieht wenn dies für kurze Zeit nicht der Fall ist – den schließlich kann man aus unerwartete Veränderungen des Verbrauchs nicht beliebig schnell reagieren.

\subsection{Leistungs-Frequenz-Regelung}
Zunächst betrachten wir die Wirkleistung in einem Drehstromnetz, dass von Kraftwerken mit dampfgetriebenen Turbinen gespeist wird. Das Kraftwerk kann dabei ein Kohle-, Öl- oder Kernkraftwerk sein. \footnote{Statt diesen dampfgetriebenen Turbinen lassen sich auch Wasser- oder Windkraftwerke annehmen, lediglich Turbinenlose Kraftwerke wie Photovoltaikanlagen sind hierfür ungeeignet} % Gaskraftwerke?
Neben den Kraftwerken gibt es auch zahlreiche Drehstrommotoren auf der \q Verbraucherseite\q. % Anführungszeichen
In den Turbinen, Generatoren und Motoren rotieren Massen – darin ist Rotationsenergie gespeichert.
Steigt nur der Wirkleistungsverbrauch über die Menge an produzierter Wirkleistung, so wird die Energiedifferenz aus der Rotationsenergie der Maschinen entnommen.
Dadurch wird deren Rotation jedoch langsamer – was zu einer Verringerung der Netzfrequenz führt.
Ist der Wirkleistungsverbrauch geringer als die Produktion, so geht die überschüssige Energie in Rotationsenergie der Maschinen über und die Netzfrequenz steigt.
In den Kraftwerken erkennt nun ein Sensor die Änderung der Frequenz und regulieren die Dampfventile entsprechend:
ist die Frequenz zu niedrig, wird mehr Dampf in die Turbinen gelassen, ist die Frequenz zu hoch, wird die Dampfzufuhr reduziert.
Man hält dadurch die Frequenz immer ungefähr konstant: in Großbritannien beträgt die Frequenz im Allgemeinen $50 \mathrm{Hz} \pm 0,05 \mathrm{Hz}$.\cite{Harrison} %Wert für Deutschland
%Details; auch in DE; alternativen?

\section{Vor- und Nachteile der HGÜ gegenüber DHÜ}



\section*{bla}
\begin{circuit}{2}
\R1 {$1 \Omega$} h
\- 8 r
\C1 C d
\- 4 v
\nl\R1 {L} d 
\nl\- 10 v
\end{circuit}

\begin{circuit}{0}
\npn1 {?} B l
\frompin npn1C
\- 1 u
\nl\A1 {$I_C$} u
\atpin npn1B
\- 1 l
\R1 {510 kohm} l
\end{circuit}


\bibliography{lit}{}
\bibliographystyle{plain}

\end{document}
