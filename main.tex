\documentclass[10pt,a4paper,twoside]{article}
\usepackage[utf8]{inputenc}
\usepackage[german]{babel}
\usepackage{amsmath}
\usepackage{amsfonts}
\usepackage{amssymb}


\author{Michael F. Schönitzer}
\title{HGÜ}
\begin{document}
\maketitle

\section{Geschichtliches}

\section{Physikalische Grundlagen}
\subsection{Ohmscher Verlust}
Fließt ein Strom $I$ mit der Spannung $U$ durch einen ohmschen Widerstand, so wird dabei die elektrische Leistung
$P_V = U' \cdot I = R \cdot I^2$
in Wärme umgesetzt – wobei $U'$ die an dem Widerstand abfallende Spannung ist.
Da die übertragene Leistung $P=U \cdot I \cdot \cos(\phi)$ ist, folgt für die Verlustleistung
\[P_V = \left(\frac{P}{U \cdot \cos\phi}\right)^2\cdot R\]
Bei der Übertragung von Energie stellt das Kabel einen Widerstand dar und die Verlustleistung soll möglichst minimiert werden. Da der Widerstand nur bedingt verringert werden kann, erhöht man die Spannung und verringert somit den Stromfluss. Eine Leitung mit einer Spannung von 110\,kV hat einen Verlust von etwa 6\% je 100\,km, eine Leitung mit 800\,kV verliert auf der selben Distanz nur etwa 0.5\%.

\section{Vor- und Nachteile der HGÜ gegenüber DHÜ}

\end{document}